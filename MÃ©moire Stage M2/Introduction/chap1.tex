\fancyhead[LO, RE]{L'EHESS}

L'\acrfull{ehess} est un grand établissement français réunissant des chercheurs et des étudiants travaillant dans les différents domaines des sciences sociales dans le monde entier. Elle est particulière et ne suit pas le schéma habituel des universités françaises, notamment par son modèle de formation par la recherche et par son ouverture internationale.

\section{Histoire de l'EHESS}
En 1868 est créée l'\acrfull{ephe} qui se compose de quatre sections (mathématiques, physique/chimie, histoire naturelle et physiologie, sciences historiques et philologiques), auquel se rajoute une cinquième section en 1886, consacrée aux sciences religieuses.

Le 3 novembre 1947, par un décret et dans le cadre d'un renouvellement institutionnel d'après-guerre, une sixième section, ancêtre de l'\acrshort{ehess}, est créée, réservée aux sciences économiques et sociales. Elle est due à trois hommes en particulier~: Lucien Febvre, Fernand Braudel et Charles Morazé, trois historiens français, qui ont par la suite été soit directeur de l'École, soit directeur d'études. L'objectif est d'édifier un établissement d'enseignement supérieur qui se démarque du champ universitaire habituel, en créant des séminaires et des centres de recherches où il est possible d'explorer tous les domaines des sciences sociales.

La section se développe au fil des années et continue dans son innovation. Les études historiques et sociologiques sont accrues et en même temps se lance un large programme d'études des sciences historiques, économiques et sociales au sein de différentes aires culturelles. C'est ainsi que se créent vers le milieu des années 50 les premiers centres de recherche et de documentation sur la Chine, l'Inde, la Russie, l'Afrique et l'Islam.

Au début des années 70, la VIe section trouve enfin un local, en s'installant au 54 boulevard Raspail dans la Maison des sciences de l'Homme, projet développé par Braudel et Febvre.

C'est par un décret du 25 janvier 1975 que la VIe section de l'\acrshort{ephe} devient finalement l'École des Hautes Études en Sciences Sociales, après que l'idée d'une émancipation de la VIe section pour permettre un meilleur développement de son orientation particulière ait été lancée par Jacques Le Goff, président de la section depuis 1972.

Deux décrets complètent le statut de l'\acrshort{ehess}~:
    \begin{itemize}
        \item Décret du 17 juillet 1984~: l'\acrshort{ehess} devient un \og~grand établissement~\fg{}, soit un établissement public à caractère scientifique, culturel et professionnel~;
        \item Décret du 12 avril 1985~: l'\acrshort{ehess} a son statut fixé dans le cadre de ses missions, ses structures et son organisation.
    \end{itemize}
L'\acrshort{ehess} est aujourd'hui implantée à Paris (siège de l'École) mais aussi à Marseille, Lyon et Toulouse. Elle s'implantera également au Campus Condorcet (en construction) où un bâtiment sera spécifiquement destiné aux équipes de l'\acrshort{ehess}.

\section{Les missions de l'EHESS}
L'\acrshort{ehess} s'organise autour d'un domaine, d'une approche ou d'une \og~aire culturelle~\fg{}, à l'instar de l'Océanie ou les mondes musulmans, par le biais de 35 unités de recherche, où peuvent s'étudier l'histoire, la sociologie, l'anthropologie, l'économie, la philosophie, la géographie, les études littéraires, la psychologie ou les sciences cognitives, à travers des séminaires favorisant l'échange entre enseignants et étudiants.

En plus d'une ouverture au monde extérieur par le biais de la recherche, l'\acrshort{ehess} s'ouvre également via de nombreux partenariats et coopérations (65 conventions de partenariat avec des universités et établissements de recherche étrangers). Près de la moitié des étudiants en master et doctorats sont étrangers et en retour, l'École dédie une grande partie des ressources pour permettre à ses membres d'aller sur le terrain et d'explorer eux même les différents lieux des aires culturelles sur lesquelles ils se spécialisent.

L'\acrshort{ehess} a pour finalité un diplôme unique (diplôme de l'\acrshort{ehess}) qui permet d'intégrer une formation de haut niveau en sciences humaines et sociales sans aucun prérequis universitaire et au-delà de l'enseignement supérieur et de la recherche, la formation de l'\acrshort{ehess} peut mener à diverses carrières, dans le journalisme, la communication, l'édition, la coopération internationale ou encore la culture.