\fancyhead[LO, RE]{État du projet}

Avant d'entamer tout travail, il est essentiel d'observer l'état du projet, à savoir son avancée actuelle mais également les limites qui se posent déjà.

\section{Un projet à ses débuts}
\subsection{Une amorce de travail}
Le projet MetaLEX\index{Projet MetaLEX} a été déterminé en termes de composition, de sources qui seront utilisées et de l'objectif final qui est attendu. Plusieurs réflexions ont été lancées pour déterminer la manière dont le travail pourrait être effectué et sur les recherches et résultats qui devront être présentés pour mener à bien le projet. Cependant, même si des décisions ont été prises au sein des membres du projet et offrent une bonne idée de ce qui est attendu, cela ne représente que sa partie théorique et d'un point de vue pratique, MetaLEX\index{Projet MetaLEX} est à peine commencé. Le travail à effectuer pour le stage a principalement pour objectif de préciser ces réflexions et d'apporter des résultats qui permettront de déterminer plus profondément la démarche future du projet. 

Si MetaLEX\index{Projet MetaLEX} n'en est effectivement qu'à une ébauche, un premier travail avait tout de même été accompli auparavant par certains membres du projet, dans le cadre de l'atelier de février. Pour cet évènement, un choix avait été fait de sélectionner un chapitre en particulier dans l'ouvrage de Beccaria\index{Traite des delits et des peines@Traité des délits et des peines}, qui sera notre source. À partir de ce chapitre, une transcription a été extraite, dans plusieurs langues, de nombreuses informations ont été récupérées et des analyses ont été réalisées. Ainsi, ces premières réflexions apportent dès lors une vision d'ensemble de ce que sera notre travail mais surtout la source d'analyse et certaines des informations récoltées développent un environnement de travail qui était précédemment vide. 

Par conséquent, grâce à ce travail, le stage peut commencer avec une base d'étude déjà établie, qu'il sera nécessaire d'étoffer et de manipuler pour obtenir une source satisfaisante pour les analyses plus avancées que nous chercherons à faire.

\subsection{Détermination de l'outil principal de travail}
Le projet MetaLEX\index{Projet MetaLEX} permet donc d'avoir déjà à disposition une source de travail dès le début de stage et cela n'est pas le seul élément précédemment arrêté pour pouvoir entamer le travail. Une réflexion avait également été menée sur le type d'étude qui devra être réalisé à partir des textes de notre corpus et comme cela a été expliqué lors de la présentation des objectifs du stage, le travail d'étude du corpus se portera sur une analyse textuelle. Il est donc nécessaire de s'intéresser aux outils à portée pour développer cette étude, à savoir des outils dédiés à la lexicométrie et à la textométrie\index{Textometrie@Textométrie}. Il en existe un certain nombre, qui peuvent être des outils généralistes, comme \textsc{r}, qui sera utilisé dans notre démonstration, bien que cela ne soit pas notre outil de travail principal, mais également des outils plus précis, tel que \textsc{iramuteq}, \textsc{lexico}, \textsc{hyperbase} ou \textsc{txm}\footcite{explorer_corpus_textuel}. C'est ce dernier qui a été choisi par les membres du projet et qui sera majoritairement notre logiciel de travail pendant la durée du stage, puisqu'il sera utilisé pour de simples observations mais également pour mettre en place l'alignement\index{Alignement}, notre objectif final.

Plateforme logicielle open-source développée dans le cadre d'un projet de textométrie\index{Textometrie@Textométrie}, \textsc{txm}\footcite{txm_plateforme} est un outil qui \og~fait évoluer la tradition lexicométrique dans un contexte nouveau\footcite[p.~219]{explorer_corpus_textuel}~\fg{},  par des corpus enrichis et structurés et un développement ouvert et collaboratif. Il est assez simple d'utilisation, que cela soit pour l'import des corpus ou pour la recherche et il offre de nombreux moyens pour des résultats concluants et hétéroclites. De plus, le logiciel peut s'installer sur les trois systèmes d'exploitation principaux (Windows, Mac, Linux), un manuel est disponible pour expliquer les différentes fonctionnalités et la plateforme est dotée d'une communauté d'utilisateurs animée par le biais de listes de diffusion et d'un site wiki, qui permettent d'apporter des détails et des précisions sur certains thèmes liés à \textsc{txm}, lorsque cela est nécessaire\footcite[p.~219-220]{explorer_corpus_textuel}.

\textsc{txm} est donc un logiciel qui offre de nombreux outils et fonctionnalités adaptés à notre étude, ce qui nous permettra d'effectuer de multiples recherches, sur différents chapitres de la source, afin de trouver les informations qui nous intéressent selon la recherche et la forme du corpus donné au logiciel, de manière à faciliter la réalisation de notre objectif.

\subsection{Un objectif à préciser}
L'objectif de notre travail, comme cela a été expliqué lors de la présentation du stage, est de réaliser un alignement\index{Alignement} avec les textes que nous avons à disposition dans notre corpus. Cependant, cela ne représente qu'une vision globale du travail. L'alignement\index{Alignement} consiste en une analyse de plusieurs textes entre eux pour observer les changements entre ces versions et si les textes que nous analyserons sont déjà déterminés, il est nécessaire de régler certains détails avant de réaliser cet alignement\index{Alignement}, ce qui n'a pas été complètement fait par les membres du projet MetaLEX\index{Projet MetaLEX} au début du stage. 

L'alignement\index{Alignement} nécessite une structure de travail, il a besoin de se voir déterminer une échelle d'analyse et ce qui doit être pris en compte pour effectuer le travail. En effet, un texte peut être découpé en plusieurs unités~: cela peut être une phrase, un paragraphe ou même les lignes selon lesquelles ce texte est segmenté. Il est donc nécessaire de décider suivant laquelle de ces structures l'alignement\index{Alignement} s'exécutera. Le cas de la ligne pourrait être pris en compte dans notre cas puisque la transcription à disposition se présente par des courtes lignes, dues à la structure dans le manuscrit. De ce fait, si le texte est conservé ainsi, l'alignement\index{Alignement} suivra cette structure, ce qui ne donnera pas le même résultat qu'un alignement\index{Alignement} par phrase ou par paragraphe. Si la décision de la structure exacte de l'alignement\index{Alignement} n'a pas été arrêtée, il a tout de même été décidé que pour les analyses, la structure exacte de la transcription, suivant le manuscrit, ne serait pas gardée et qu'il sera nécessaire de modifier la mise en forme du texte pour que ce dernier soit segmenté seulement par des phrases et des paragraphes. 

Une autre interrogation s'est posée au sujet de cet alignement\index{Alignement}~: sa base. Il s'agit de savoir si l'alignement\index{Alignement} concerne tout le texte ou si l'intérêt doit se porter sur des mots en particulier. Le projet MetaLEX\index{Projet MetaLEX} étant concentré sur l'évolution d'un lexique juridique, il est pertinent de se demander si cela doit être considéré comme le fondement de l'alignement\index{Alignement} pour observer les remplacements, suppressions et insertions ou s'il faut s'intéresser aux changements du texte dans son intégralité. 

Ces divers détails représenteront la base de la réflexion qui s'effectuera lors des analyses de la source, de manière à décider avec l'expérience du travail sur le corpus la meilleure manière de représenter cet alignement\index{Alignement}, notamment avec les moyens qui seront mis à disposition pour le concevoir et l'obtenir.

\paragraph{}Ainsi, l'avancée actuelle du projet, même si elle n'est qu'à une ébauche, possède déjà certains éléments qui permettent de mettre en place une réflexion sur le travail à venir, sur ce qu'il faudra effectuer pour le stage, pour mener à bien les tâches requises. Ces éléments sont néanmoins des problèmes mineurs, puisque l'avancée du travail de stage pourra résoudre les interrogations. Cependant, il existe d'autres problèmes, dépendants du projet, qui ne seront pas solutionnés et empêcheront une étude exhaustive du corpus.

\section{Des limites inhérentes au projet}
Deux limites principales s'imposent au sein du projet et empêcheront la réalisation d'un travail complet~: la première est due à mes propres compétences et la seconde à la matérialité du corpus.

\subsection{Non exhaustivité du travail}
Notre corpus de travail se compose de multiples éditions de la source principale du projet MetaLEX\index{Projet MetaLEX}, le \emph{Traité des délits et des peines}\index{Traite des delits et des peines@Traité des délits et des peines}, que nous présenterons par la suite. Ces éditions ont la particularité d'être présentes en quatre langues différentes~: la langue d'origine, l'italien, puis le français, l'allemand et l'anglais. Notre travail se déroulera en deux grandes étapes ne prenant pas en compte les mêmes critères pour leur réalisation. L'une des étapes se base sur des éléments établis et ne divergera peu ou pas en fonction des langues. Une compréhension écrite n'est donc pas nécessaire lors de ce travail. L'autre étape cependant concerne l'analyse de textes, et bien que certaines statistiques ne demandent pas une compréhension écrite de la langue, la majorité requiert d'avoir un entendement des textes à disposition pour permettre d'effectuer un travail complet et précis. C'est dans ce cas que le travail s'interrompra pour une partie du corpus. 

Le corpus se divise en quatre langues~: ma compréhension écrite du français et de l'anglais est excellente, ce qui me permet d'effectuer toutes les analyses et toutes les recherches nécessaires pour ces deux parties du corpus sans difficulté~; je n'ai jamais étudié l'italien mais j'en ai tout de même une compréhension écrite minimale, l'italien faisant partie des langues romanes, comme le français et l'espagnol, dont j'ai également une compréhension écrite basique et étant hérité du latin, dont j'ai des connaissances de base. Il est donc aisé, avec l'aide de traducteurs et de dictionnaires, de passer outre les incompréhensions et de pouvoir analyser le texte. L'allemand représente, pour moi, la véritable limite dans l'étude du corpus. Bien que cela soit une langue germanique comme l'anglais, il possède certaines caractéristiques qui m'empêchent de l'étudier pour le stage. Tout d'abord, l'alphabet n'est pas tout à fait le même, puisqu'en plus de quelques voyelles utilisées souvent avec un tréma, il possède une lettre en plus, le \textit{ß}, considéré comme un \og~s pointu~\fg{}, qui ne s'utilise que dans certains cas particuliers. Ensuite, les textes en allemand se présentent dans un style d'écriture spécial, le \textit{Fraktur}, une écriture gothique majoritairement utilisée en allemand, qui rend encore plus difficile la compréhension écrite des documents. Enfin, bien que son origine soit la même que l'anglais et que certains mots peuvent parfois être proches, le vocabulaire est généralement très particulier et n'a pas de correspondances avec d'autres langues, comme cela avait pu être le cas pour l'italien. Avec toutes ces différentes conditions réunies, il m'était impossible de travailler jusqu'au bout sur le corpus allemand. C'est pourquoi le travail sur le corpus se fait pour les quatre langues jusqu'à l'étape de la correction orthographique que nous exposerons dans la seconde partie. À partir de cette étape, l'étude ne s'effectue que sur l'anglais, le français et l'italien, à part pour quelques analyses statistiques courtes et non approfondies.

\subsection{Un corpus incomplet}
Le corpus, comme nous l'avons vu, se compose de nombreuses éditions qui seront toutes manipulées et exploitées afin d'être analysées. Seulement, l'un des obstacles présents au début du stage est l'absence d'un certain nombre de ces textes. Dans le cadre de l'atelier de février, le chapitre 30/31/36 (choisi comme chapitre de base pour l'atelier et numéroté différemment en fonction de l'édition) de toutes les éditions qui pouvaient être disponibles à la numérisation pour cet évènement avait été transcrit et fait partie de la source d'analyse pour le stage. Cependant, l'un de nos objectifs est l'augmentation de ces sources d'analyse, ce qui sera ralenti par l'absence d'un certain nombre de ces éditions, pourtant répertoriées pour le chapitre 30/31/36, car non numérisées en entier et maintenant difficiles à acquérir dans un temps imparti.

Les numérisations de bibliothèques peuvent parfois être difficiles et assez longues~; la recherche d'ouvrages non encore trouvés peut s'avérer encore plus ardue et il est impossible, dès le début du stage, d'avoir un corpus complet pour l'étude. Il est donc probable que dans le cas de certains travaux au sein même du projet et du stage, certaines informations fassent défaut. Bien qu'au fur et à mesure du stage, de nouvelles sources apparaîtront, le temps de travail sur chacune est relativement long, ce qui explique qu'elles ne puissent pas figurer tout au long du processus. C'est ce qui expliquera la présence pour quelques analyses de certains ouvrages du corpus qui ne figureront pas par la suite, en fonction, par exemple, de l'année ou de la langue.

\paragraph{}Pour conclure, nous pouvons affirmer qu'exception faite de quelques limites qui ralentiront le travail pendant le stage, le projet est déjà bien établi, avec des objectifs posés et des moyens de les remplir à disposition. Si le problème de la langue est un vrai obstacle en termes d'exhaustivité du travail, le manque de certaines éditions pour l'étude peut paraître dérisoire proportionnellement au nombre d'éditions à disposition, notamment lorsque nous nous intéressons plus en détail à la source et à ses origines.