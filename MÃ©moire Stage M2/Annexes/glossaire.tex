\chapter{Glossaire}
\fancyhead[LO, RE]{Glossaire}

Le glossaire contient les définitions des termes clés de notre projet, notamment un lexique portant sur l'étude statistique de texte, ainsi que certains termes plus obscurs que nous pouvons retrouver dans le corps du texte.

\paragraph{Alignement\index{Alignement}} : Méthode visant à analyser deux ou plusieurs versions d'un même texte pour rechercher les différences et les invariants existants.

\paragraph{Analyse Factorielle des Correspondances (\acrshort{afc})} : Méthode qui permet d'étudier l'association entre deux variables qualitatives. Elle sert à décrire et hiérarchiser des relations statistiques.

\paragraph{Collation} : Comparaison d'exemplaires manuscrits ou imprimés entre eux.

\paragraph{Concordancier} : Logiciel qui travaille des chaînes de caractères au sein d'un texte et permet de placer, pour toutes ses occurences, un mot recherché dans le contexte droit et gauche de la phrase où il se trouve.

\paragraph{Cooccurrence} : Présence simultannée de mots ou groupes de mots dans un même extrait de taille restreinte.

\paragraph{\og Débruitage \fg{}} : Élimination du \og bruit \fg{} dans une image numérisée, c'est-à-dire éliminer les pixels noirs qui se sont rajoutés dans les zones de pixels blancs, ce qui peut engendrer une déformation de certains caractères ou des éléments dans les zones de lecture qui perturbent la reconnaissance de caractères. Ce bruit peut avoir pour origine un problème lors de la numérisation ou être dû à l'état de la page du manuscrit, qui peut avoir des ratures ou des détériorations.

\paragraph{Géomatique} : Ensemble des outils et méthodes permettant d'acquérir, de représenter, d'analyser et d'intégrer des données géographiques. 

\paragraph{Lemmatisation} : Identification d'un mot sous sa forme canonique (le lemme représente le singulier d'un nom, le masculin singulier d'un adjectif, l'infinitif d'un verbe, etc.).

\paragraph{Lexicométrie} : Science linguistique étudiant statistiquement l'utilisation des mots.

\paragraph{Métalexicographie} : Discipline qui étudie les méthodes et les principes guidant la création des dictionnaires.

\paragraph{Reconnaissance Optique de Caractères (en anglais, Optical Character Recognitition (\acrshort{ocr}))\index{OCR}} : Procédé permettant de transformer les images d'un texte, dactylographié ou manuscrit, en des fichiers électroniques au format texte.

\paragraph{Statistique textuelle\index{Statistique textuelle}} : Outil mêlant à la fois statistique classique, linguistique, analyse du discours et informatique, qui permet d'étudier des textes en utilisant les méthodes statistiques.

\paragraph{Textométrie\index{Textometrie@Textométrie}} : Discipline développée principalement dans les années 1970 qui se base sur les méthodes d'analyse des données appliquées à des données linguistiques et textuelles. Elle met en avant des modèles statistiques qui rendent compte de caractères significatifs de texte : attirances contextuelles des mots, linéarité et organisation interne du texte, contrastes intertextuels ou indicateurs d'évolution lexicale. La textométrie\index{Textometrie@Textométrie} met donc en avant un large éventail de calculs et de statistiques pour analyser des collections de textes.

\paragraph{Traitement Automatique des Langues (\acrshort{tal})}: Discipline qui associe linguistes et informaticiens. Son objectif est de développer des logiciels ou programmes informatiques capables de traiter automatiquement des données linguistiques, en prenant en compte les spécificités du langage humain. Les principaux domaines sont le traitement de la parole, la traduction automatique, la compréhension automatique des textes, la gestion électronique de l'information et des documents existants et la génération automatique de textes.