\chapter{Chronologie du stage}
\fancyhead[LO, RE]{Chronologie du stage}

Cette chronologie reprend les évènements clés du stage et du projet auquel je participe, raison pour laquelle elle s'étend avant et après la période de stage. Elle comprend à la fois les séminaires auxquels j'ai assisté et les réunions de projet que nous avons eu régulièrement avec les autres membres de MetaLEX\index{Projet MetaLEX}. Je n'y ai cependant pas cité toutes les brèves réunions que j'ai eue avec mon encadrante tout au long du stage.

\paragraph{19-22 février 2019} : Séminaire sur \og Les mots du droit \fg{}, premières approches sur le futur contenu du stage, sur les enjeux et premières discussions sur le travail à effectuer (les membres du projet MetaLEX\index{Projet MetaLEX} étaient presque tous présents).

\paragraph{1er avril 2019} : Début du stage.

\paragraph{11 avril 2019} : Réunion avec Falk Bretschneider, Rainer Kiesow et Carmen Brando pour présenter les premiers travaux effectués, la prise de connaissance du corpus et de la tâche à effectuer, discussion sur ce qui sera attendu pour la suite du projet, sur ce qui devra être fait, choix des deux futurs extraits à océriser\index{OCR} pour continuer le travail.

\paragraph{16 mai 2019} : Séminaire Labex \acrshort{obvil} \og Statistiques textuelles \fg{}  à la Maison de la Recherche, présenté par Florian Cafiero, sur les statistiques textuelles (spécificités et implications).

\paragraph{28 mai 2019} : Réunion à la plateforme avec Falk, Rainer et Carmen et en Skype avec les membres trêvois pour discuter de l'avancée du projet après deux mois de stage, pour répondre aux questions et problèmes qui se posaient, pour apporter des modifications à certains aspects des travaux déjà réalisés pour correspondre plus à l'enjeu général et pour décider de la marche à suivre pour les deux mois restants.

\paragraph{12 juin 2019} : Réunion avec Falk et Carmen, présentation des méthodes réalisées pour mettre en place le dictionnaires de termes juridiques pour l'annotation des corpus et les premiers résultats sur \textsc{txm}.

\paragraph{19 juin 2019} : Atelier \acrshort{tal} \og Méthodes et pratiques de la statistique textuelle avec R \fg{} à la Plateforme Géomatique.

\paragraph{1er juillet 2019} : Skype avec l'équipe du projet MetaLex\index{Projet MetaLEX} pour discuter de l'avancée du travail au bout de trois mois, présenter/expliquer le tableau d'alignement commencé pour répondre à la problématique du stage et réfléchir aux moyens à utiliser pour approfondir le travail d'annotation et d'alignement fait avec \textsc{txm}.

\paragraph{8 juillet 2019} : Vidéoconférence avec Cristof Schöch, membre du projet MetaLEX\index{Projet MetaLEX} et Serge Heiden, responsable du projet \textsc{txm}, pour discuter de l'application du corpus avec \textsc{txm}, des moyens de perfectionner l'alignement en utilisant l'outil de textométrie\index{Textometrie@Textométrie} et des étapes à réaliser pour possiblement créer des éditions synoptiques avec le corpus de Beccaria.

\paragraph{16 juillet 2019} : Réunion avec Falk et Carmen pour évoquer l'étendue du travail déjà effectué et ce qui doit encore être fait pour les deux semaines restantes, apprendre ce qui se déroulera pendant l'atelier de la Villa Vigoni d'octobre et ce qu'il faudra présenter.

\paragraph{31 juillet 2019} : Fin du stage.

\paragraph{7-9 octobre 2019} : Atelier à la Villa Vigoni (Loveno di Menaggio) dans le cadre du projet \og MetaLEX - Métalexicographie des langues du droit \fg{}\index{Projet MetaLEX} et financé par le Centre Interdisciplinaire d’études et de Recherches sur l’Allemagne (CIERA), qui est un réseau de recherche interdisciplinaire et international qui favorise et soutient la coopération scientifiques entre la France et l'Allemagne.  Présentation de ma participation au projet pendant les quatre mois de stage et des avancées réalisées.