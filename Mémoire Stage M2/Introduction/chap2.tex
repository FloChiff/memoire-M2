\fancyhead[LO, RE]{Projet MetaLEX}

Mon stage s'inscrit dans le cadre d'une collaboration entre le Centre Georg Simmel de l'\acrshort{ehess} et le Trier Center for Digital Humanities, avec le soutien de la plateforme géomatique et humanités numériques de l'\acrshort{ehess}, pour le projet \og~MetaLEX - Dictionnaire historique du droit et des institutions et pratiques judiciaires~\fg{} \index{Projet MetaLEX}.

\section{Les membres et collaborateurs du projet}
\subsection{Le Centre Georg Simmel}
Parmi les 35 unités de recherches de l'\acrshort{ehess} se trouve le centre Georg Simmel, situé au 54, boulevard Raspail dans le 4ème arrondissement, qui est une unité mixte, partagée avec le CNRS et qui est issue du Centre de recherches interdisciplinaires sur l'Allemagne datant de 2001.

Spécialisés dans les recherches franco-allemandes en sciences sociales, ses chercheurs s'intéressent à un certain nombre de problématiques concernant de nombreuses disciplines (histoire, anthropologie, droit, études littéraires et germaniques, etc.) pour penser le monde en transformation. Ces recherches franco-allemandes se placent ainsi dans une approche européenne et transnationale et se développent par le biais de débats méthodologiques et théoriques, en lien avec les disciplines mentionnées précédemment.

Les recherches s'axent sur quatre grands points~: actes de la création~; travail, capacité et parcours biographiques~; fabriques de la frontière~; effet des langues (herméneutique, épistémologie, historicités). Ces grands axes sont développés par les divers membres du Centre.

\subsection{Le Trier Center for Digital Humanities}
Appartenant à la faculté de Langage, Littérature et études des Médias de l'université de Trêves, le Centre d'Humanités numériques est un centre international de recherche et de service, soutenu notamment par le programme de l'université de Rhineland Palatinate~: "Wissen schafft Zukunt" (\og~La science crée le futur~\fg{}).

Par le biais d'études interdisciplinaires, le Centre tend à soutenir et à apporter des développements dans les humanités numériques, comme dans le cadre de dictionnaires numériques, éditions et sources primaires. 

\subsection{La Plateforme géomatique et en humanités numériques de l'EHESS}
S’appuyant sur l’animation et la direction de plusieurs ingénieurs, la Plateforme géomatique et en humanités numériques dispose de moyens humains et matériels, de ressources, d’outils de stockage, de services et d’un site internet. Ces éléments sont sollicités pour répondre à des besoins grandissants dans l’accompagnement de projets de recherche en sciences humaines et sociales avec un volet numérique mobilisant des connaissances et des techniques en sciences de l'information géographique, traitement et analyse de données textuelles, modélisation de données historiques, dans le développement d’outils informatiques ou encore dans la formation continue et ponctuelle des étudiants et chercheurs.

La Plateforme travaille à réaliser ces objectifs par le biais de nombreux partenariats, dont neuf centres de l’\acrshort{ehess} tel que le Centre de Recherches Historiques (CRH), porteur historique du projet plateforme. Elle collabore également avec des membres de Paris Sciences Lettres (PSL) et du Campus Condorcet comme le Centre national de la recherche scientifique (CNRS) ou l’École des Chartes (ENC).

\section{Le projet MetaLEX}
Le projet MetaLEX\index{Projet MetaLEX} ou \og~Métalexicographie numérique des langues historiques du droit en Europe~\fg{} est un projet développé par Falk Bretschneider, maître de conférence en Histoire à l'\acrshort{ehess}, Rainer Maria Kiesow, directeur d'étude à l'\acrshort{ehess} et spécialisé en droit, Carmen Brando, ingénieure de recherche en humanités numériques à l'\acrshort{ehess}, ainsi que Christof Schöch, professeur en humanités numériques, Claudine Moulin, professeur de linguistique historique, Vera Hildenbrandt et Thomas Burch, issus de l'Université de Trêves.

\subsection{Objectifs du projet}
L'objectif du projet est d'élaborer un système d'informations métalexicographiques, concernant le vocabulaire portant sur les langues historiques du droit en Europe. Il doit retracer l'évolution des notions historiques d'un lemme et non d'un mot en particulier. Le projet s'intéresse non pas à des mots isolés mais à un lexique, pour le situer ensuite dans un contexte spatio-temporel et pouvoir ainsi renouveler la recherche sur les interconnexions et interdépendances de notions appartenant au vocabulaire juridique européen et caractérisant les sources qui l'alimentent.

Le rendu final devrait être une plateforme de recherches interdisciplinaires sur les langues historiques du droit en Europe. Cela permettra d'observer les différentes traductions de termes juridiques entre plusieurs langues. L'objectif est d'offrir la possibilité d'avoir un accès rapide et direct à la diversité lexicale du vocabulaire juridique dans le contexte spatio-temporel des langues juridiques nationales.

Le projet n'étant encore qu'à ses débuts, les sources utilisées pour le lexique se basent sur une période de deux cents ans, entre 1700 et 1900, à travers des sources multilingues européennes. L'objectif ensuite serait d'enrichir la plateforme avec des données issues d'une période plus élargie et un nombre plus important de sources, pour qu'au final, le projet puisse servir, sur un plus long terme et dans une forme définitive, à établir le multilinguisme juridique actuel en Europe et ce que cela suscite.

\subsection{Enjeux du projet}
L'étude sur le vocabulaire des langues historiques du droit en Europe est un concept qui regroupe trois disciplines (histoire, droit et linguistique) et pourtant, elle n'est que très peu documentée. La recherche linguistique pour des notions juridiques est plutôt restreinte et de plus, elle se fait généralement par langue, sans aucune volonté d'études transversales pour observer les différences ou les liens entre un même mot de droit dans différentes langues. En outre, dans les cas des études sur les termes juridiques, cela est majoritairement limité à une présentation du mot en tant que tel. Il n'y a eu aucune interrogation continue sur son histoire, son origine, son évolution et son impact. Le mot n'est pas étudié en profondeur et c'est donc une documentation superficielle, que le projet MetaLEX\index{Projet MetaLEX} vise à améliorer.

Le projet a donc pour enjeu de fournir une étude sur le vocabulaire juridique, sans se limiter à une seule langue et en prêtant attention à certains détails pour offrir une histoire transculturelle et linguistique de ces concepts et donner le moyen, à l'aide du numérique, d'avoir facilement accès à ces concepts, grâce à une plateforme qui ne devrait pas seulement permettre de trouver rapidement l'origine et l'histoire d'un mot mais également de le situer dans un contexte spatio-temporel.