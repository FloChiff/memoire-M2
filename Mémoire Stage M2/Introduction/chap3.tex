\fancyhead[LO, RE]{Le stage}

S'inscrivant dans une participation au projet MetaLEX\index{Projet MetaLEX}, le stage, qui se déroule sur une période de quatre mois, a pour but de contribuer au projet par la réalisation de plusieurs tâches, afin d'élargir les connaissances sur le corpus sélectionné et d'établir plus précisément ce qu'il faudra effectuer pour le mener à bien.

Le stage sollicite des connaissances informatiques et littéraires. Les connaissances informatiques se basent sur une aptitude au traitement de texte, avec de l'\acrshort{ocr}\index{OCR}, un langage de programmation et une compétence en traitement automatique des langues (\acrshort{tal}), mais aussi à l'analyse textuelle, avec un travail de textométrie\index{Textometrie@Textométrie} et d'alignement de textes\index{Alignement}. Les connaissances littéraires se concentrent sur de l'histoire et de la linguistique. Le stage porte en grande partie sur l'étude d'un texte juridique, donc un attrait pour l'histoire du droit est sollicitée, dans le but de comprendre l'importance du document qui sera à notre disposition et surtout l'enjeu qu'il a représenté lors de sa publication. L'aptitude linguistique s'appuie surtout sur la diversité du corpus et sa présence en quatre langues différentes, ce qui nécessite donc une compréhension écrite minimale pour trois langues, en plus du français (anglais, italien et allemand).

Suivant cela et en utilisant ces connaissances, le stage requiert l'accomplissement de trois tâches liées entre elles, pour approcher de la réalisation des objectifs du projet MetaLEX\index{Projet MetaLEX}. Tout d'abord, l'étude des éditions et traductions du Beccaria doit aboutir à la mise en place d'une généalogie des traductions du texte, de manière à établir une hiérarchie entre les différentes traductions, les changements effectués, les modifications dans les chapitres, etc. Ensuite, il est nécessaire d'effectuer des transcriptions de texte à partir des ouvrages qui ont été étudiés puis hiérarchisés durant la première étape. Ces transcriptions permettront d'avoir une source d'analyse à disposition, à l'aide d'un processus d'\acrshort{ocr}\index{OCR} et d'un nettoyage subséquent des textes avec des outils de \acrlong{tal}. Enfin, nous irons exploiter ces nouveaux textes afin d'atteindre l'objectif ultime du stage, c'est-à-dire l'alignement\index{Alignement} des éditions et des traductions du \emph{Traité\index{Traite des delits et des peines@Traité des délits et des peines}}, en produisant une procédure semi-automatique pour cet alignement\index{Alignement}. \pagebreak

Une première ébauche de certaines de ces tâches avait été réalisée à l'occasion d'un séminaire intitulé \og~Les mots du droit. Lexicographie numérique de \emph{Des délits et des peines} de Cesare Beccaria et ses traductions en Europe~\fg{} et organisé par les membres du projet MetaLEX\index{Projet MetaLEX} en février 2019. Afin d'effectuer diverses manipulations pendant cet atelier, une procédure d'océrisation\index{OCR} avait été mise en place et un chapitre avait été transcrit à partir des différentes éditions disponibles. Un tableau de ces éditions avait également été mis en place, contenant une réflexion sur la généalogie des textes, des éléments d'analyses avaient été extraits et les premières idées sur la mise en forme qui sera adoptée pour les textes avaient été lancées. Ce séminaire a préparé en amont une partie de mon travail et m'a aidé à mieux cerner les objectifs du projet et les réflexions qui les accompagnent.

Ainsi, cela nous permet d'observer que pour permettre la réalisation de ces objectifs, il est essentiel de s'interroger sur les éléments qu'il faudra prendre en compte, les techniques à exploiter et les logiciels à utiliser, de même que les étapes à établir pour atteindre ce résultat. Par conséquent, la problématique de ce stage sera de savoir quelles manipulations et quelles analyses devront être effectuées sur la source à disposition pour parvenir à produire un alignement\index{Alignement} des éditions et des traductions.

Nous répondrons à cela en trois parties. Nous nous intéresserons tout d'abord à l'observation des éléments déjà à disposition lors du début du stage~; puis, nous entreprendrons la transcription et le traitement de plusieurs chapitres des différentes éditions du corpus~; enfin, nous travaillerons à l'exploitation, de nombreuses manières, de ces nouvelles transcriptions pour chercher, au final, à concevoir un processus d'alignement\index{Alignement} et à créer ce qui en résultera~: l'alignement\index{Alignement!alignement cible@alignement ciblé} ciblé des éditions et des traductions à partir des termes juridiques qui ponctuent le texte.