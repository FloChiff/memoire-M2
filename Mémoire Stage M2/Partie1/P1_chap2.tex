\fancyhead[LO, RE]{L'\oe{}uvre de Beccaria}

Le travail pendant le stage porte sur un traité de droit~: un document juridique majeur pour le droit pénal, le \emph{Traité des Délits et des Peines}\index{Traite des delits et des peines@Traité des délits et des peines} écrit par Cesare Beccaria\index{Beccaria, marquis de} et publié pour la première fois en 1764.  

\section{Genèse de l'\oe{}uvre}
Né en 1738 à Milan, Cesare Bonesana, dit marquis de Beccaria\index{Beccaria, marquis de}, est issu de la noblesse~: il est Visconti du coté de sa mère, et il a eu le titre de marquis de son père. Il suit une éducation classique de huit ans et obtient le grade de docteur en droit à 20 ans. Il se rebelle rapidement contre l'autorité familiale et contre l'autorité politique et sociale.

Tout d'abord, Cesare Beccaria\index{Beccaria, marquis de} tombe amoureux d'une femme venant d'une famille \og~inférieure~\fg{} et veut l'épouser, ce qui le met en opposition contre son père. Il est assigné à résidence trois mois, son père espérant qu'il oublie la jeune femme, sans résultat. Il épouse donc Theresa Blanco, ce qui entraîne son renvoi du domicile paternel et la fin de son accès à la fortune familiale.

Ensuite, le contexte politique et social compliqué de Milan, marqué par la guerre contre l'Autriche, la fatigue et la mort, le font réagir. Au milieu de nombreuses réformes de religion et d'éducation, auquel il est favorable, il crée avec deux frères un cercle philosophique~: l'académie \emph{Dei pugni} ou \og~Académie des coups de poing~\fg{}. Ils s'intéressent tous trois alors à la philosophie des Lumières, les écrits et les pamphlets et notamment aux questions judiciaires et aux problèmes liés à la criminalité et sa répression, qui est un sujet proche d'eux, compte tenu du travail d'un des deux frères, Alessandro Verri, \og~protecteur~\fg{} des prisons à Milan. Les frères feront parler d'eux après la sortie du \emph{Traité}\index{Traite des delits et des peines@Traité des délits et des peines}, puisqu'il y a eu quelques débats d'authenticité quant à l'auteur de l'œuvre, Alessandro Verri s'étant proclamé comme l'auteur du \emph{Traité}\index{Traite des delits et des peines@Traité des délits et des peines}, celui-ci ayant été publié anonymement la première fois.

Ce contexte dans lequel Beccaria vit le pousse ainsi à publier, à l'âge de 25 ans, le \emph{Traité des délits et des peines}\index{Traite des delits et des peines@Traité des délits et des peines} avec lequel il se rebelle contre son père et contre le système, grâce à un texte révolutionnaire.

\section{Le \emph{Traité des délits et des peines}}
En 1764, sort anonymement le \emph{Traité des Délits et des Peines}\index{Traite des delits et des peines@Traité des délits et des peines}, ou \emph{Dei Delitti e delle Pene}\index{Traite des delits et des peines@Traité des délits et des peines} dans son italien d'origine.

Le texte cible le fanatisme religieux et le barbarisme juridique et évoque de très nombreux thèmes en une centaine de pages. Il mentionne des éléments déjà remis en question par la philosophie des Lumières tels que l'arbitraire des pratiques et des procédures juridiques ou la peine de mort mais aussi des sujets beaucoup plus choquants  pour l'époque, tels que l'homosexualité et sa dépénalisation. Avec son traité\index{Traite des delits et des peines@Traité des délits et des peines}, il cherche à présenter un nouvel ordre à mettre en place, pour une société plus juste, en organisant notamment un système de peines qui correspondent aux délits, pour éviter le cas par cas qui avait lieu à l'époque. En effet, l'époque moderne ne disposait d'aucun code pénal unique et les magistrats jugeaient à l'aide de normes ou de textes anciens datant de deux ou trois siècles avant, sans établir des règles qui seraient réutilisées par la suite. Beccaria recommande notamment d'alléger certaines peines, beaucoup trop extrêmes parfois, au regard de la gravité du délit, notamment la peine de mort et la torture, qui devraient être réservées aux cas de crimes vraiment graves. Il propose également des alternatives à la peine de mort, tel que le travail forcé qui, s'il a la même finalité, permet un apport économique à la société, les prisonniers coûtant moins cher, selon le courant utilitariste auquel adhère Beccaria.

Ainsi, en seulement une centaine de pages, ce qui est assez extraordinaire pour l'époque, le marquis de Beccaria réussit à véhiculer un très grand nombre d'idées révolutionnaires, qui entraîneront un essor de l'ouvrage et un véritable retentissement dans l'Europe des Lumières.

\section{Portée du traité en Europe}
\subsection{Un succès à travers toute l'Europe}
L'ouvrage rencontre un grand succès très rapidement tout d'abord à Milan et dans les régions alentours où l'italien est la langue vernaculaire, puis dans le reste de l'Europe, ce qui s'observe par de nombreuses traductions~: française (1765), allemande (1765), anglaise (1767), suédoise (1770), polonaise (1772) ou encore espagnole (1774). Ensuite, le traité\index{Traite des delits et des peines@Traité des délits et des peines} fait l'objet de très nombreuses rééditions, agrémentées de commentaires et de préfaces par de grands auteurs, tels que Voltaire et Diderot pour les versions françaises et certaines traductions subséquentes dans d'autres langues, avant même la fin du siècle. Certains iront jusqu'à reprendre le texte, changer sa disposition et la manière dont le contenu des chapitres est présenté, comme André Morellet\index{Morellet, Andre@Morellet, André}, qui traduit et édite l'ouvrage en français en 1766, car, comme il le justifie~: \og~Nous en avions le droit~; parce qu'un Livre où l'on plaide si éloquemment la cause de l'Humanité, appartient désormais au Monde et à toutes les Nations~\footcite{fr1-1}\fg{}. Beccaria lui-même réédite son ouvrage plusieurs fois, en incluant des nouveaux chapitres ou des commentaires, la version définitive pouvant être considérée comme celle de 1766\footcite{it6}.

\subsection{La place du \emph{Traité} dans le droit pénal européen}
Le traité\index{Traite des delits et des peines@Traité des délits et des peines} est publié pendant deux procès judiciaires scandaleux, l'affaire Calas \footnote{\url{http://www.justice.gouv.fr/histoire-et-patrimoine-10050/proces-historiques-10411/laffaire-calas-22774.html}} et celle du chevalier de la Barre \footnote{\url{https://francearchives.fr/de/commemo/recueil-2016/39461}}. 
Dans l'affaire Calas, un dénommé Jean Calas, protestant, est condamné à mort (roué vif, étranglé et brûlé) pour avoir prétendument tué son fils (qui s'est suicidé), car il aurait voulu se convertir au catholicisme. 
Dans l'affaire du chevalier de la Barre, tout commence avec un crucifix tailladé sur un pont. Le lieutenant en charge de l'affaire en veut personnellement à la famille de la Barre et inculpe le chevalier sous les faux prétextes de chansons irréligieuses et de ne pas s'être découvert devant une procession religieuse. Les magistrats condamnent à mort le chevalier de 19 ans et malgré les demandes de l'Église d'une clémence royale, Louis XV refuse d'alléger la peine et le chevalier est exécuté.

Ces deux affaires judiciaires, qui se sont déroulées entre 1761 et 1766, sont la preuve même d'erreurs judiciaires et de la nécessité d'un modèle de délits et de peines à adapter entre elles, ce qui est en corrélation avec les idées présentées par Beccaria et explique d'autant plus le retentissement et l'intérêt pour les principes qu'il expose.

Le succès du \emph{Traité}\index{Traite des delits et des peines@Traité des délits et des peines} s'observe surtout par les multiples réformes judiciaires qu'il entraîne dans plusieurs pays d'Europe, notamment vis-à-vis de la torture et la peine de mort. Il peut être considéré comme étant à l'origine de la pensée juridique moderne, étant encore aujourd'hui mentionné comme un des courants majeurs du droit pénal. Ce \emph{Traité}\index{Traite des delits et des peines@Traité des délits et des peines} est à l'origine de codes pénaux et d'amendements, comme en Suède, en Russie et aux États-Unis et certains des principes énoncés dans cet ouvrage seront inscrits comme droit de l'homme et du citoyen après la Révolution française.

\section{Le \emph{Traité} aujourd'hui~: matérialité et obtention de la source}
Le traité\index{Traite des delits et des peines@Traité des délits et des peines} a donc eu une grande influence sur l'Europe et même en Amérique et a eu des conséquences sur le monde juridique et sur le droit pénal pendant de nombreuses décennies après sa parution. Dans le contexte actuel, il est encore édité en plusieurs langues et mentionné pendant les cours de droit pénal et son étude est notamment intéressante car c'est un ouvrage très disponible, à la fois dans de nombreuses langues et dans ses versions les plus anciennes.

\subsection{État physique de la source}
Notre travail consiste à étudier les diverses versions du \emph{Traité}\index{Traite des delits et des peines@Traité des délits et des peines} de Beccaria et il est donc nécessaire pour cela de sélectionner des éditions. Le choix a été fait de se limiter à celles produites avant le XIXème siècle, ce qui correspond à l'enjeu du projet MetaLEX\index{Projet MetaLEX}. Il est essentiel alors de trouver des informations sur toutes les éditions du \emph{Traité}\index{Traite des delits et des peines@Traité des délits et des peines} produites entre 1764 et 1800 et dans les quatre langues choisies~: italien, français, allemand, anglais.

Le travail, fait en amont par les membres du projet et supporté par les recherches de Philippe Audegean\footcite{beccaria_audegean_2009}, apporte un peu plus d'une trentaine d'éditions, réparties plus ou moins équitablement entre chaque langue (une dizaine pour l'italien, l'allemand et le français, cinq pour l'anglais). Cette première recherche fournit déjà certains détails à propos des évolutions du \emph{Traité}\index{Traite des delits et des peines@Traité des délits et des peines}~: les différences entre les versions de l'abbé Morellet\index{Morellet, Andre@Morellet, André} et du marquis de Beccaria apparaissent~; les premières éditions en italien sont soit des contrefaçons, soit des améliorations apportées par Beccaria. Plus précisément encore, les recherches montrent, par exemple, qu'en allemand, la toute première édition du \emph{Traité}\index{Traite des delits et des peines@Traité des délits et des peines}, parue en 1765, a disparu et n'existe plus aujourd'hui~; en français, une édition parue en 1782 ne fournit pas le traité\index{Traite des delits et des peines@Traité des délits et des peines} comme un ouvrage unique mais le place au sein d'un ouvrage plus générique intitulé \emph{Bibliothèque philosophique du législateur du politique, du jurisconsulte}\footcite{fr2-2}.

\begin{landscape}
\pagestyle{empty}
\begin{longtable}{|c|c|c|c|c|c|c|}
\caption{Éditions du \emph{Traité des délits et des peines} de Beccaria d'avant 1800}
\endhead
\hline
sigle & année & traducteur & base & réédition & lieu, éditeur \\ \hline
it1 & 1764 & - & - &  & anonyme \\ \hline
it2* & 1764 & - & - &  & Monaco (Florence) \\ \hline
it3 & 1765 & - & - &  & Lausanne (faux) \\ \hline
it4* & -1764 & - & - &  & Monaco (Pise ou Livourne  ?) \\ \hline
it5 & 1766 & - & - &  & Harlem (Livourne), Coltelini \\ \hline
it6 & 1766 & - & - &  & Harlem \\ \hline
it7 & 1774 & - & fr1-1 &  & Londres (Livourne) \\ \hline
it8 & 1780 & - & fr1-1 &  & Harlem \\ \hline
it9 & 1781 & - & fr1-1 &  & Venice, Benvenuti \\ \hline
it10 & 1786 & - & fr1-1 &  & Paris, Cazin \\ \hline \hline
all1-1 & 1765 & Butschek &  &  & Prague \\ \hline
all2-1 & 1766 & Wittenberg & fr1-1 &  & Hambourg, Bock \\ \hline
all3-1 & 1767 & Schultes & it5  ? &  & Ulm, Bartholomäi \\ \hline
all3-2 & 1776 & Schultes & it5  ? & X & Tyrnau \\ \hline
all4-1 & 1767 & Montag  ? & fr1-1, it5  ? &  & Prague, Clauser \\ \hline
all4-1 & 1778 & Flathe & it5  ? &  & Breslau, Korn \\ \hline
all4-2 & 1786 & Flathe & it5  ? & X & Vienne, Trattner \\ \hline
all5-1 & 1788 & ? & it9 &  & Breslau, Korn \\ \hline
all6-1 & 1798 & Bergk & ? &  & Leipzig, Beygang \\ \hline
\pagebreak \hline ang1-1 & 1767 & anonyme & it5  ? &  & Dublin, Exshaw \\ \hline
ang1-2 & 1769 & anonyme & it5  ? &  & Londres, Newberry \\ \hline
ang1-3 & 1778 & anonyme & it5  ? &  & Édimbourg, Donaldson \\ \hline
ang1-4 & 1785 & anonyme & it5  ? &  & Londres, Newberry \\ \hline
ang1-5 & 1788 & anonyme & it5  ? &  & Édimbourg, Donaldson \\ \hline \hline
fr1-1 & 1766 (1765) & Morellet & it3 &  & Lausanne (Paris), Roederer \\ \hline
fr1-2 & 1797 & Morellet & it3 & X & Paris, Imprimerie du journal d’économie \\ \hline
fr2-1 & 1773 & Chaillou de Lisy & ? &  & Paris, Bastien \\ \hline
fr2-2 & 1782 & Chaillou de Lisy & ? & X &  \\ \hline
fr2-3 & 1784 & Chaillou de Lisy & it6 & X & Paris, Pichard \\ \hline
fr2-4 & 1794 & Chaillou de Lisy & ? & X & Paris, Martin \& Gauthier \\ \hline
fr2-5 & 1796 & Chaillou de Lisy & ? & X & Paris, Boiste \\ \hline
fr2-7 & 1797 & Chaillou de Lisy & ? & X & Neuchâtel \\ \hline
fr3-1 & 1794 & Maine de Biran & ? &  &  \\ \hline
\end{longtable}
\label{table:editions}
\end{landscape}

Une fois que cette liste d'éditions a été établie et qu'un maximum de recherches ont été faites pour trouver des informations sur l'état de ces sources aujourd'hui, il est primordial de s'intéresser à son obtention, pour pouvoir travailler dessus aisément.

\subsection{Méthodes d'acquisition des éléments du corpus}
Une des étapes les plus compliquées lorsque le travail se porte sur un ancien manuscrit, datant de plusieurs siècles auparavant, est de retrouver cette source dans un bon état de conservation pour directement travailler avec. Notre cas est d'autant plus particulier que nous sommes intéressés par l'ouvrage dans sa multitude et si cet ouvrage n'est pas disponible dans ses diverses versions, le travail sera d'autant plus ardu. Favorablement pour notre étude, la majorité des éditions retrouvées lors des recherches sont disponibles encore aujourd'hui et sur deux supports~: manuscrit et numérique. Dans ce second cas, il est en effet possible de retrouver un certain nombre des éditions du \emph{Traité}\index{Traite des delits et des peines@Traité des délits et des peines} de Beccaria déjà numérisées et accessibles en majorité gratuitement sur Google Books ou sur des plateformes de bibliothèques qui mettent à disposition le \textsc{pdf} en téléchargement. Pour les éditions qui ne sont pas disponibles directement sous ce format, elles sont présentes dans des bibliothèques, dans un état de conservation correcte et d'assez bonne qualité pour qu'une numérisation puisse être possible et ainsi avoir accès au document pour le soumettre à l'analyse et pouvoir le manipuler sans difficulté. 

Parmi toutes les éditions du \emph{Traité}\index{Traite des delits et des peines@Traité des délits et des peines} de Beccaria trouvées pendant les recherches, la majorité fait partie du corpus qui servira à notre travail pendant le stage. Les éditions italiennes ont toutes été retrouvées mais ne sont pas toutes à disposition (les contrefaçons de 1764 ne sont pas accessibles immédiatement). Les éditions allemandes sont toutes accessibles en format numérique dans notre corpus, sauf pour la première édition allemande, disparue. Les éditions françaises ont deux éditions non disponibles~: une présente à la BnF et à Berlin, mais non encore numérisée et une, datant de 1794, non retrouvée. Enfin, les cinq éditions anglaises d'avant 1800 retrouvées sont à disposition dans le corpus. 

\paragraph{} Ainsi, le \emph{Traité des délits et des peines}\index{Traite des delits et des peines@Traité des délits et des peines} du marquis de Beccaria\index{Beccaria, marquis de} a été un texte d'un très grand retentissement, qui a marqué l'histoire juridique et a eu une immense portée à travers l'Europe. Il a entraîné la production de nombreuses éditions, avant même le XIXe siècle, éditions qui sont encore aujourd'hui pour la plupart disponibles et nous permettent ainsi d'effectuer une analyse à plus grande échelle, grâce à leur multitude. Maintenant que ces sources sont à disposition, nous allons donc avoir la possibilité de les exploiter pour maximiser l'efficacité de notre travail et faciliter notre analyse, qui sera plus étendue grâce à la multitude de textes.